\documentclass[usletter]{article}
\usepackage[latin1]{inputenc}
%\usepackage[french]{babel}
%\usepackage{t1enc}
%\usepackage[francais]{babel}

\usepackage{vmargin}
\usepackage{amssymb,amstext,amsmath}
\usepackage{hyperref}
\usepackage{epsfig}
\usepackage{array}
\usepackage{color}
\usepackage{xspace}

\usepackage{latexsym}

\usepackage{tikz} 

\usepgflibrary{shapes.misc} 

\newcommand{\cmd}[1]{\texttt{#1}}

%\newtheorem{command}{Command}{\bfseries}{\normalfont}

\renewcommand{\baselinestretch}{1.5}

\title{RevBayes -- Phylogenies and the comparative method}


\begin{document}
\maketitle

\section*{Introduction}

The subject of the comparative method is the analysis of trait evolution at the macroevolutionary scale.
Traits can be morphological, behavioral or can be related to the life-history strategies (e.g. longevity).
In a comparative context, many different questions can be addressed: tempo and mode of evolution, correlated evolution of multiple quantitative traits, trends and bursts, changes in evolutionary mode correlated with major key innovations in some groups, etc.

In order to correctly formalize comparative questions, the underlying phylogeny should always be be explicitly accounted for. This point is clearly illustrated, in particular, by the independent contrast method (ref Felsenstein). Practically speaking, this is most often done by first estimating the phylogeny and the divergence times using a separate phylogenetic reconstruction software. In a second step, this time-calibrated phylogeny is used as an input to the comparative method.
Doing this, however, raises a certain number of methodological problems:
\begin{itemize}
\item
the uncertainty about the phylogeny (and about divergence times) is ignored
\item
the traits themselves may have something to say about the phylogeny
\item
the rate of substitution, and more generally the parameters of the substitution process, can also be seen as quantitative traits, amenable to a comparative analysis.
\end{itemize}
All these points are not easily formalized in the context of the step-wise approach mentioned above.
Instead, what all this suggests is that phylogenetic reconstruction, molecular dating and the comparative method should all be considered jointly, in the context of one single overarching probabilistic model.

The modular structure of RevBayes makes it a natural framework for attempting this integration.
The aim of the present tutorial is to guide you through a series of examples where this integration is achieved, step by step.
It can also be considered as an example of the more general perspective of \emph{integrative modeling}, which can be recruited in many other contexts.

\section*{Data and files}

In the \cmd{data} folder, you will find the following files
\begin{itemize}
\item
\cmd{plac40lhtlog.nex}: 3 life-history traits (age at sexual maturity, body mass, maximum recorded lifespan) for 40 placental mammals. The traits have been log-transformed.
\item
\cmd{plac40\_4fold.nex}: an alignment made of a concatenation of 17 nuclear genes in 40 placental mammals (only the four-fold degenerate third coding positions)
\item
\cmd{chronoplac40.tree}: a time-calibrated phylogeny, which has been obtained by running another software program.
On the cluster, and if you are logged under an X-terminal, you can visualize this tree using the \cmd{njplot} command (ref):
\\
\cmd{njplot chronoplac40.tree}
\end{itemize}

\section{Correlated evolution of traits using a fixed phylogeny}

We would like to estimate the correlation between the $K=3$ life-history traits, such as given in the \cmd{plac40lhtlog.nex} file, and this, while properly taking into account phylogenetic inertia. 
To do so, we will assume that the traits jointly evolve along the phylogeny as a \emph{multivariate Brownian process}.
We will estimate the \emph{covariance matrix} of this process. The empirical support in favor of positive or negative correlations between pairs of traits will then be formalized in terms of posterior probabilities of having positive or negative entries in this covariance matrix.
As a by-product of this correlation analysis, we will also obtain a marginal ancestral reconstruction of the traits along the entire phylogeny, which we will then visualize using graphical software programs.

In a first step, we will ignore phylogenetic uncertainty.
Thus, we will assume that the Brownian process describing trait evolution will run along a fixed phylogeny, such as specified in the file \cmd{chronoplac40.tree}.

\subsection*{The model and the priors}

A multivariate Brownian process $X(t)$, of dimension $K$ (here $K=3$),
is entirely parameterized by its starting value ($X(0)$ at the root of the phylogeny) and a $K \times K$ covariance matrix, which we will call $\Sigma$. A positive entry between two traits, say $\Sigma_{12} > 0$, means that when trait 1 increases, trait 2 also tends to increase. Conversely, a negative entry means that the two traits tend to undergo variation in opposite directions.
As for the diagonal entries (e.g. $\Sigma_{11}$), they represent the variance per unit of time (i.e. the speed of evolution) of each trait considered marginally.
%In a Bayesian context, the empirical support in favor of the existence of a postive (resp\. negative) correlation between 2 traits can therefore be obtained by measuring the posterior probability that the corresponding entry of $\Sigma$ is positive (resp\. negative). Using Monte Carlo method, we will obtain sampled from the joint posterior distribution

On $\Sigma$, we will assume an inverse-Wishart prior:
\begin{eqnarray*}
\Sigma &\sim& W^{-1}(\Sigma_0, d)
\end{eqnarray*}
The inverse Wishart distribution has two parameters: a symmetric matrix $\Sigma_0$ of same dimension as $\Sigma$, and a natural number $d \ge K+1$ (the number of degrees of freedom). Roughly speaking, the inverse Wishart prior is centered on $\frac{1}{d} \Sigma_0$ and is all the more concentrated around this center than $d$ is large.
The choice of this prior is primarily motivated by the fact that it is analytically conjugate to the multivariate normal distribution and, therefore, is very convenient when used for the covariance matrix of a (log-normal) Brownian motion.

Since we want a diffuse prior, we will use a small value for $d$, e.g. $d=K+2$.
In addition, we want the prior to be centered on a diagonal matrix (i.e. we want to be a priori \emph{equi-poised} with respect to either positive or negative correlations among traits). We may also want our prior to be invariant by linear reparameterizations of traits (linear transformations on the logarithm of life-history traits are equivalent to allometric re-parameterizations of the traits considered in natural units). This invariance will hold if and only if $\Sigma_0$ is proportional to the identity matrix, i.e. if $\Sigma_0 = \kappa I_K$, for some positive real number $\kappa$. This number $\kappa$ will set the amplitude of the variation per unit of time of the traits. Since we have no idea about the scale of $\kappa$, it is therefore natural to use a \emph{scale-invariant}, or \emph{log-uniform} prior (due to Jeffreys):
\begin{eqnarray*}
\kappa &\sim& 1/\kappa
\end{eqnarray*}
Note that, in the present context, where we don't know anything about the \emph{scale} of $\kappa$, a uniform prior on $\kappa$ would not have been optimal (why?).
Note also that the log-uniform prior is improper. We can make it proper by truncating it (e.g. $\kappa$ is log-uniform over 12 orders of magnitude, that is, over $[10^{-6}, 10^{6}]$).

Concerning the Brownian process $X(t)$, we still have to define a prior on its starting value $X(0)$ at the root of the phylogeny. We may want this prior to be invariant upon changing the units in which the traits are measured, so as to be insensitive to the fact that, for example, body mass has been given in kilograms, and not in grams. In that case, our prior should be invariant by translation, i.e. should be improper uniform. Currently, this is the only prior implemented in RevBayes for the starting value of a multivariate Brownian process running along a phylogeny.

Finally, the tree topology is, as mentioned above, fixed to some externally given phylogeny. However, in order to fully specify the model in RevBayes, we will still need to define a prior on this tree: here, we will simply use a uniform prior, both on the topology and on divergence times.

With this, the entire model is specified: tree $\tau$, scale $\kappa$, covariance matrix $\Sigma$ and Brownian process $X(t)$:
\begin{eqnarray*}
\tau &\sim& \text{Uniform}
\\
\kappa &\sim& \text{Log-uniform}
\\
\Sigma &\sim& W^{-1}(\Sigma_0 = \kappa I_K , \, d = K+2)
\\
X &\sim& \text{Brownian}(\tau, \, \Sigma)
\end{eqnarray*}
Constraining $\tau$ based on an externally-provided phylogeny and conditioning the model on empirical data by clamping $X(t)$ at the tips of the phylogeny to the values provided in the trait data matrix, we can then run a MCMC to sample from the joint posterior distribution on $\kappa, \Sigma$ and $X$. Once this is done, we can estimate marginal posterior probabilities (e.g. for positive or negative covariance among traits) in terms of frequencies at which entries of the matrix have been sampled as positive or negative. We can also obtain posterior means, medians or credible intervals for the value of body mass or other life-history traits for specific ancestors along the phylogeny.

\subsection*{Programming the model in RevBayes}

Just doing what has been explained in the last subsection, but now in the RevBayes language:
\begin{itemize}
\item
load trait data:
\\
\cmd{contData <- readCharacterData("data/plac40lhtlog.nex")}
\item
get number of traits, number of taxa and taxon names:
\\
\cmd{
nTraits <- contData.nchar()[1]
\\
nTaxa <- contData.ntaxa()
\\
names <- contData.names()
}
\item
define a uniformly-distirbuted time-tree:
\\
\cmd{tau $\sim$ uniformTimeTree(originTime = 1.0, taxonNames = names)}
\item
set this tree equal to the one defined in \cmd{chronoplac40.tree}:
\\
\cmd{treeArray <- readTrees("data/chronoplac40.tree")
\\
fixedTree <- treeArray[1]
\\
tau.setValue(fixedTree)}
\item
define $\kappa$:
\\
\cmd{kappa $\sim$ logUniform(1e-6, 1e6)}
\item
define the number of degrees of freedom as $d = K+2$:
\\
\cmd{df <- nTraits+2}
\item
define the covariance matrix $\Sigma$ as inverse Wishart:
\\
\cmd{sigma $\sim$ invWishart(dim=nTraits, kappa=kappa, df=df)}
\\
note that we are using here a special constructor for the inverse Wishart: when a natural number and a positive real number are given, it is implicitly assumed that the matrix-valued parameter is $\kappa I_dim$.
\item
define the multivariate Brownian process:
\\
\cmd{x $\sim$ mvtBrownian(tau,sigma)}
\item
condition the Brownian model on quantitative trait data,
i.e. clamp the process at the tips of the tree, at the values observed in extant taxa,
such as given by \cmd{contData}. This needs to be done trait by trait:
\\
\cmd{
for (i in 1:nTraits)    \{
        br.clampAt(contData,i,i)
\}
}
\\
Here, we give twice the index \cmd{i}: the first corresponds to the entry of the Brownian process, and the second one to the column of the data matrix. In some cases (as we will see below), the Brownian process and the data matrix may not be of same dimension, and therefore, it will be useful to be able to specify arbitrary maps between them.
\end{itemize}
The model is now entirely specified. We can define the moves on its parameters:
\begin{itemize}
\item
initialize a running index for storing moves:
\\
\cmd{index <- 1}
\item
push a scaling move on $\kappa$:
\\
\cmd{
moves[index] <- mScale(kappa, lambda=2.0, tune=true, weight=3.0)
\\
index <- index + 1
}
\item
a sliding move on the Brownian process
\\
\cmd{
moves[index] <- mvMultivariatePhyloProcessSliding(process=br,lambda=1,
\\
tune=true,weight=100)
\\
index <- index + 1
}
\item
a global translation move on the Brownian process: this move applies a random translation across the entire phylogeny, for one trait taken at random among the $K$ taits:
\\
\cmd{
moves[index] <- mvMultivariatePhyloProcessTranslation(process=br,lambda=0.1,
\\
tune=true,weight=1)
\\
index <- index + 1
}
Note that the translation move concerns the entire phylogeny: therefore, its tuning parameter is smaller than that used for the single-node sliding move above.
\item
finally, a conjugate Gibbs move for $\Sigma$: as it turns out, conditional on $\kappa$ and the Brownian process $X$, it is possible to directly resample $\Sigma$ from its conditional posterior distribution (see ref Lartillot Poujol). In RevBayes, this is implemented as follows:
\\
\cmd{
moves[index] <- mvConjugateInverseWishartBrownian(sigma=sigma, process=br,
\\
kappa=kappa, df=df, weight=1)
\\
index <- index + 1
}
\end{itemize}
Before creating the model, we need to define a few summary statistics, which we want to track during MCMC, either to monitor convergence or for obtaining interesting outputs.
First, suppose you are specifically interested in the covariance and the correlation coefficient associated with the joint variation of body-size (trait 2) and longevity (trait 3). You may also be interested in the \emph{partial} correlation coefficient between body mass and longevity, i.e. while controlling for variation in age at sexual maturity. These three quantities can be singled out and named as follows:
\begin{itemize}
\item
the covariance:
\\
\cmd{
cov23 := sigma.covariance(2,3)
}
\item
the correlation coefficient:
\\
\cmd{
cor23 := sigma.correlation(2,3)
}
\item
the partial correlation:
\\
\cmd{
parcor23 := sigma.partialCorrelation(2,3)
}
\item
note that the variance per unit of time of, say, log body mass is given by the diagonal entry:
\\
\cmd{
var2 := sigma.covariance(2,2)
}
\item
we can also get all correlation coefficients as a vector:
\begin{verbatim}
corrindex <- 1
for (i in 1:nTraits)    {
    for (j in i+1:nTraits) {
        correl[corrindex] := sigma.correlation(i,j)
        corrindex <- corrindex + 1
    }
}
\end{verbatim}
\item
we could be interested in tracking several summary statistics also for the Brownian motion, in particular the mean and the standard deviation along the tree, separately for each trait:
\begin{verbatim}
for (i in 1:nTraits)    {
        meanbr[i] := X.mean(i)
        stdevbr[i] := X.stdev(i)
}
\end{verbatim}
\end{itemize}
After creating the model, all these new variables (\cmd{cor12}, \cmd{correl}, \cmd{meanbr}, etc) can be monitored, along with $\Sigma$, $X$ and $\kappa$:
\begin{itemize}
\item
create the model
\\
\cmd{
mymodel <- model(kappa)
}
\item
make a screen monitor that tracks correlation coefficients and mean Brownian values:
\\
\cmd{
monitors[1] <- screenmonitor(printgen=10, correl, meanbr)
}
\item
a file monitor for $\Sigma$ and the correlation coefficients:
\\
\cmd{
monitors[2] <- filemonitor(filename="output/plactraits.cov",printgen=10, separator = "  ", sigma, correl)
}
\item
a file monitor for the ancestral reconstruction of traits:
\\
\cmd{
monitors[3] <- filemonitor(filename="output/plactraits.traits",printgen=10, separator = "       ", X)
}
\item
and a general model monitor:
\\
\cmd{
monitors[4] <- modelmonitor(filename="output/plactraits.log",printgen=10, separator = " ")
}
\end{itemize}
We can finally create a mcmc, burn it in and run it for a good 100 000 cycles:
\\
\cmd{
mymcmc <- mcmc(mymodel, monitors, moves)
\\
mymcmc.burnin(generations=100,tuningInterval=100)
\\
mymcmc.run(100000)
}


\begin{itemize}
\item
take the 40-taxon placental dataset, with 3 life-history traits (body mass, longevity and sexual maturity) taken from the AnAge database (ref).
\item
fit a Brownian motion of dimension 3
\item
estimate covariance matrix. Identify diagonal elements with the variance parameter of the one-dimensional Brownian motion fitted previously.
\item
output posterior distribution on: covariance parameters, correlation coefficients
\item
output partial correlation coefficient between, say, maturity and longevity when controlling for body size.
\end{itemize}

\section*{Accounting for uncertainty in phylogeny and divergence times}

The models implemented in the last section are structured as follows:
\begin{itemize}
\item
a data set of quantitative traits is loaded
\item
a phylogenetic tree is declared, whose tips correspond to the taxa of the data set
\item
the phylogenetic tree is clamped to an externally given tree topology with divergence times specified
\item
a comparative model is declared and conditioned on the quantitative trait data
\item
running the model gives posterior distributions over the parameters
\item
marginalization on the parameters of interest (covariance or correlation coefficient) essentially answers to our original question (are traits correlated)
\end{itemize}

Starting from this model, how could we account for phylogenetic uncertainty?
One possibility would be as follows.
Typically, the tree given as an input is a consensus tree obtained by running a Bayesian software program (like Beast or RevBayes) on another data set, made of aligned genetic sequences for the same set of taxa.
Thus, instead of running the comparative analysis only on the consensus tree, one could instead run it on a series of, say, 100 or 1000 trees sampled from the posterior distribution (after burn-in). Then, the resulting estimates could be pooled and averaged. However, doing this is a bit problematic.

Another possibility is to make a hierarchical model where
\begin{itemize}
\item
two data sets are loaded: one for sequence data and one for quantitative traits
\item
A tree is declared, with a prior (birth death or uniform)
\item
a substitution model is declared, parameterized by the tree
\item
the substitution model is clamped on the molecular data.
\item
a comparative model is declared, parameterized by the same tree
\item
the comparative model is clamped on the quantitative trait data
\item
the model is run
\end{itemize}
Here, the important conceptual difference with the previous analysis the tree will be inferred using both sources of empirical data. In practice, this is not likely to make such a big difference (usually, sequence data have much more signal than quantitative traits for informing the phylogeny). But this is conceptually more elegant and, more importantly, will be the basis for further modeling developments.

\emph{Would be useful to draw some graphical model representations of the various models alluded to in this section: essentially, how to merge together two disconnected graphical models, one representing the comparative method and another one representing the phylogenetic analysis into one single connected graph.}

\section{Autocorrelated relaxed molecular clock}

\emph{
Should make a more explicit connection with 
the tutorial on molecular dating, in which tools were already presented for conducting a robust dating analysis.}

In the previous model, no consideration was made for the problem of rate variation among lineages. This is of course problematic, in particular at the phylogenetic scale considered here (mammals), where we know that there is substantial rate variation. 

In addition, we know that substitution rates are auto-correlated in the present case: typically, entire orders, such as rodents, are fast evolving, whereas other orders like Cetartiodactyla, are slowly evolving. In other words, nearby lineages along the phylogeny, which they tend to belong to the same order, also tend to be characterized by similar substitution rates.


The auto-correlated clock is fundamentally a model where the log of the instant substitution rate is seen as a Brownian motion. Thus, it is exactly like a univariate quantitative trait, such as those that we have modeled in the first section of this tutorial. Since the Brownian motion describes the evolution of the log of the rate, we need to exponentiate this brownian process in order to obtain substitution rates, which can then be plugged into the substitution model.
\begin{itemize}
\item
load sequence data
\item
declare the tree
\item
define a univariate Brownian motion along this tree, using the same tools as for the univariate comparative analysis
\item
exponentiate the Brownian process, using exponentialBranchTree. This will create a deterministic function of this brownian motion, summarized through branchwise averages.
\item
plug these rates into the substModel object, as the branchRates parameter
\item
condition the model and run the program.
\end{itemize}
You may compare this analysis with the kind of analyses that were done earlier in the context of the molecular dating session. Could also be interesting to reconstruct the evolution of substitution rate as a trait along the phylogeny, and visualize it as a color-coded tree: just to emphasize the conceptual similarity between substitution rate and quantitative traits.

Note that, here, we do not have included any fossil information: we are merely doing \emph{relative} dating. This is not ideal, but we will see at the end of this tutorial how all this can be integrated with fossil information. In fact, this represents one of the most exciting frontiers of the present integrative approach.

\section{Rates and traits}

\subsection{Merging the clock and the comparative analysis}

Once you have been able to construct this simple auto-correlated relaxed clock model, and based on what was done in the previous section, it should not be too difficult to run a joint dating and comparative analysis. Thus, you would essentially declare two Brownian motions along the same tree: one univariate motion for the relaxed clock, and one multiariate Brownian motion for the quantitative traits.

Once this is done, we can now ask one further obvious question: why considering the substitution rate and the quantitative traits as separate Brownian motions? Why not considering them as a joint multivariate motion? Doing so would have one major advantage: the correlated evolution of rates and traits will be automatically estimated, as a by-product of the model.

\section{A comparative analysis of the variation in equilibrium GC content}

\emph{Time-heterogeneous models: have they been introduced at some point in the previous tutorials?}

Some words here about the prevalence of compositional variation, and the possible causes. Discussing the alternative between (1) modeling variation in GC using iid GC parameters branchwise, and (2) modeling this variation by assuming an explicit time-continuous process for GC evolution along the lineages. The latter is more principled, and one should thoroughly discuss why.

Some exercises: 
\begin{itemize}
\item
mammalian dataset (add the karyotypic quantitative traits here): correlation between GC and recombination rate and body size. The biased gene conversion hypothesis
\item
in the case of mammals, possible to do some multiple regression analysis on gc, rate and traits.
\item
archaeal rRNA GC content and temperature.
\end{itemize}

\section{Towards integrative macro-evolution modeling}

The integration proposed here is just one example of the integration of multiple domains of macroevolutionary studies that could be done with revbayes.

\subsection{Rates and traits with fossil calibrations}

Make a calibrated analysis: either using "focal dating", based on Tracy's method. The 5 to 7 few fossil calibrations typically used for mammals could easily be recruited in this context.

\subsection{Total evidence}

Alternatively, undertake a more ambitious and more extensive total evidence approach. The obvious problem here is that the approach requires a good fossil sampling, with a good matrix of both discrete and continuous traits. Getting those data is probably a fair amount of work in itself. But the outcome could be very exciting: in particular, adding body size information about fossils!

\subsection{Testing diversification models}

make an integrated diversification studies and comparative analysis: combining phylogenetic estimation, dating, reconstruction of body size evolution and test of a diversification model in the case of placentals: in their case, a good model to test would be a skyline birth death or, perhaps more interestingly, a burst and stasis model, with the burst occurring right at KT.

\subsection{Comparative analysis and incomplete lineage sorting}

Let the generation time $t$, the mutation rate per generation $u$ and the effective population size $N_e$ be a joint multivariate (log-normal) Brownian process along the species tree. Then, plug $(N_e t)^{-1}$ as the coalescence rate within the ILS model, and $u / t$ as the substitution rate per calendar unit of time in the substitution model running along gene genealogies. Of course, all this can be correlated with body size and life-history traits. Predictions are: $N_e$ correlates negatively, $t$ positively and $u$ positively (but $u / t$ negatively) with body-size. Nice allometric analysis of ILS...

\emph{One nasty thing here is that we need to pre-define classes similar to ExponentialBranchTree, but that would implement the functional relations introduced above between $N_e$, $t$ and $u$: since, for the moment, we do not have the kind of templated containers, combined with user-defined functions, that would allow users to do that automatically...}

\section{Some notes on the overall modeling strategy used here}

\begin{itemize}
\item
emphasize that the model is still approximate here: in particular, in the way we take the average over branches. In this respect, mention recent developments in fine-grained models of Brownian evolution of substitution rates
\item
ultimately, no need to restrict oneself to Brownian motions: more general processes could be imagined.
\end{itemize}

\bibliographystyle{mbe}

\bibliography{allbib}

\end{document}  


\subsection{Further directions}

Emphasize that what is currently available in revbayes is but a small part of what can be done in the context of the comparative method. In this respect, mention the already existing software programs: BayesTraits, Ape, everything currently existing in the R environment. Also briefly mention the connections with GLS and maximum likelihood.

However, the main advantage of RevBayes is that we can now integrate the comparative method with the phylogenetic models, as will now be done.

